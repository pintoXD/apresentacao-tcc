\section{Seção I}
\begin{frame}{Explicações}
    % itemize
    Este é um template que pode ser utilizado para:
    \begin{itemize}
        \item Apresentação de Trabalhos Acadêmicos
        \item Apresentação de Disciplinas
        \item Apresentações de Teses e Dissertações
    \end{itemize}

    \vspace{0.4cm} % vertical space
    
    % enumeration
    Para utilizar este template corretamente é importante que:
    \begin{enumerate}
        \item Tenha conhecimento mínimo sobre LaTeX
        \item Ler os comentários no template (explicações)
        \item Ler o README.md (documentação)
    \end{enumerate}

    \vspace{0.2cm}

    \example{Este é um texto de exemplo!} \emph{Texto de Ênfase!}
\end{frame}

%% ---------------------------------------------------------------------------
\subsection{Subseção I}
\begin{frame}{Criando Blocos}
    % Blocks styles
    \begin{block}{Bloco Padrão}
        Texto do corpo do bloco.
    \end{block}

    \begin{alertblock}{Bloco de Alerta}
        Texto do corpo do bloco.
    \end{alertblock}

    \begin{exampleblock}{Bloco de Exemplo}
        Texto do corpo do bloco.
    \end{exampleblock}   
\end{frame}

%% ---------------------------------------------------------------------------
\subsection{Subseção II}
\begin{frame}{Criando Caixas}
    \successbox{testando o success box}

    \pause

    \alertbox{testando o alert box}

    \pause

    \simplebox{testando o simple box}
\end{frame}

%% ---------------------------------------------------------------------------
\subsection{Subseção III}
\begin{frame}{Criando Algoritmos (Pseudocódigo)}
    \begin{algorithm}[H]
        \SetAlgoLined
        \LinesNumbered
        \SetKwInOut{Input}{input}
        \SetKwInOut{Output}{output}
        \Input{x: float, y: float}
        \Output{r: float}
        \While{True}{
          r = x + y\;
          \eIf{r >= 30}{
           ``O valor de $r$ é maior ou iqual a 10.''\;
           break\;
           }{
           ``O valor de $r$ = '', r\;
          }
         } 
         \caption{Algorithm Example}
    \end{algorithm}
\end{frame}

%% ---------------------------------------------------------------------------

\begin{frame}{Inserindo Algoritmos}
    \lstset{language=Python}
    \lstinputlisting[language=Python]{code/main.py}
\end{frame}

%% ---------------------------------------------------------------------------
\begin{frame}{Inserindo Algoritmos}
    \lstinputlisting[language=C]{code/source.c}
\end{frame}

%% ---------------------------------------------------------------------------
\begin{frame}{Inserindo Algoritmos}
    \lstinputlisting[language=Java]{code/helloworld.java}
\end{frame}

%% ---------------------------------------------------------------------------
\begin{frame}{Inserindo Algoritmos}
    \lstinputlisting[language=HTML]{code/index.html}
\end{frame}

%% ---------------------------------------------------------------------------
% This frame show an example to insert multicolumns
