%%%%%%%%%%%%%%%%%%%%%%%%%%%%%%%%%%%%%%%%%%%%%%%%%%%%%%%%%%%%%%%%%%%%%
%% This file contains the packages that can be used in the beamer. %%
%%%%%%%%%%%%%%%%%%%%%%%%%%%%%%%%%%%%%%%%%%%%%%%%%%%%%%%%%%%%%%%%%%%%%
% Package to fonts family
\usepackage[T1]{fontenc}
% Package to accentuation
\usepackage[utf8]{inputenc}
% Package to Portuguese language
\usepackage[brazil]{babel}
% Package to Figures
\usepackage{graphicx}
% Package to the colors
\usepackage{color}
% Package to the colors
\usepackage{xcolor}
% Packages to math symbols and expressions
\usepackage{amsfonts, amssymb, amsmath}
% Package to multiple lines and columns in table
\usepackage{multirow, array} 
% Package to create pseudo-code
% For more detail of this package: http://linorg.usp.br/CTAN/macros/latex/contrib/algorithm2e/doc/algorithm2e.pdf
\usepackage{algorithm2e}
% Package to insert code
\usepackage{listings} 
\usepackage{keyval}
% Package to justify text
\usepackage[document]{ragged2e}
% Package to manage the bibliography
\usepackage[backend=biber, style=numeric, sorting=none]{biblatex}
% Package to facilities quotations
\usepackage{csquotes}
% Package to use multicols
\usepackage{multicol}

\usepackage{booktabs}

\usepackage[flushleft]{threeparttable}

\def\boxit#1{%
  \smash{\color{red}\fboxrule=1pt\relax\fboxsep=2pt\relax%
  \llap{\rlap{\fbox{\vphantom{0}\makebox[#1]{}}}~}}\ignorespaces
}