\section{Introdução}


\begin{frame}
    
    \centering
    \color{blue_theme}\huge{\textbf{Introdução}}

\end{frame}



\begin{frame}{Introdução}
    % itemize
    \begin{block}{Definição}
        
        Um dispositivo \textit{datalogger} é um sistema embarcado que realiza e 
        leituras de um ambiente, por meio de sensores, e mantém esses dados
        armazenados para uso futuro.
    \end{block}



    % \vspace{30pt}



\end{frame}

\begin{frame}  

    Métodos de recuperação dos dados coletados: 
    
    \begin{itemize}
        \item Manual - Um operador deve ir ao local de instalação. Preço unitário acessível;  
        \item Automatizada - Envio de informações via interface sem fio. Eleva o preço unitário do \textit{datalogger}.
    \end{itemize}

\end{frame}    
    % Demonstrar o processo de desenvolvimento do hardware de um sistema embarcado usando como exemplo a concepção e desenvolvimento de um dispositivo \textit{datalogger}
    
    % Este é um template que pode ser utilizado para:
    % \begin{itemize}
    %     \item Apresentação de Trabalhos Acadêmicos
    %     \item Apresentação de Disciplinas
    %     \item Apresentações de Teses e Dissertações
    % \end{itemize}

    % % \vspace{0.4cm} % vertical space
    
    % % enumeration
    % Para utilizar este template corretamente é importante que:
    % \begin{enumerate}
    %     \item Tenha conhecimento mínimo sobre LaTeX
    %     \item Ler os comentários no template (explicações)
    %     \item Ler o README.md (documentação)
    % \end{enumerate}

    % \vspace{0.2cm}

    % \example{Este é um texto de exemplo!} \emph{Texto de Ênfase!}
% \end{frame}

% \begin{frame}{Introdução}


% \end{frame}






\begin{frame}{Objetivo geral}
    % itemize
    Desenvolver os esquemáticos eletrônicos e leiaute da placa de circuito impresso de um \textit{datalogger} de baixo custo, com interfaces Wi-Fi e \textit{Bluetooth}, que possa realizar medições de temperatura, umidade e luminosidade. 
    
    % \begin{itemize}
    %     \item Recuperação manual de dados;
    %     \item Recuperação automatizada de dados;
    % \end{itemize}
    
    
    
    % Demonstrar o processo de desenvolvimento do hardware de um sistema embarcado usando como exemplo a concepção e desenvolvimento de um dispositivo \textit{datalogger}
    
    % Este é um template que pode ser utilizado para:
    % \begin{itemize}
    %     \item Apresentação de Trabalhos Acadêmicos
    %     \item Apresentação de Disciplinas
    %     \item Apresentações de Teses e Dissertações
    % \end{itemize}

    % % \vspace{0.4cm} % vertical space
    
    % % enumeration
    % Para utilizar este template corretamente é importante que:
    % \begin{enumerate}
    %     \item Tenha conhecimento mínimo sobre LaTeX
    %     \item Ler os comentários no template (explicações)
    %     \item Ler o README.md (documentação)
    % \end{enumerate}

    % \vspace{0.2cm}

    % \example{Este é um texto de exemplo!} \emph{Texto de Ênfase!}
\end{frame}



\begin{frame}{Objetivos específicos}

Os objetivos específicos desse trabalho são:
    
\begin{itemize}
    \item Análise de soluções existentes;
    \item Levantamento de escopo e especificações;
    \item Criação de arquitetura;
    \item Seleção de componentes e criação de esquemáticos eletrônicos;
    \item Desenvolvimento de PCI;
    \item Mensuração dos custos;
    \item Definição da autonomia típica;
    \item Comparativo de mercado.
\end{itemize}
    
    
\end{frame}
