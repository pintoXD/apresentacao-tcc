% \section{Fundamentação Teórica}

% \begin{frame}
    
%     \centering
%     \color{blue_theme}\huge{{Fundamentação}}

% \end{frame}


\begin{frame}{Sistemas Embarcados}
    \begin{block}{Definição}
        
        São sistemas computacionais que são parte integrante de 
        um produto ou ferramenta e são limitados em tamanho, consumo, 
        poder de processamento e custo. 

        
    \end{block}

    % \vfill{}

    \vspace{30pt}

    Produtos que possuem um sistema embarcado são:

    \begin{itemize}
        % \item Controles remotos;
        \item Brinquedos;
        \item Eletrodomésticos;
        \item Automóveis;
    \end{itemize}


\end{frame}

%% ---------------------------------------------------------------------------

% \begin{frame}{Sistemas Embarcados}

% \end{frame}

\begin{frame}{Sistemas Embarcados}

    Estrutura básica:
    \begin{itemize}
        \item Unidade de fornecimento de energia elétrica;
        \item Interfaces de entrada e saída para interação;
        \item Memórias de dados e de programa;
        \item Interfaces de comunicação;
        \item \textbf{Unidade de processamento};
        
    \end{itemize}
\end{frame}



\begin{frame}{Tecnologias de processadores}

    \begin{block}{Definição}
        Maneira como a unidade de processamento é 
        organizada para executar instruções.
    \end{block}


    \begin{itemize}
        \item Processadores de Uso Geral
        \item Processadores Especializados
        \begin{itemize}
            \item Microcontroladores;
            \item DSPs.
        \end{itemize}
        \item Processadores Dedicados
        \begin{itemize}
            \item ASICs
            \item FPGAs
        \end{itemize}
        \item System-On-A-Chip
    \end{itemize}



\end{frame}

% \begin{frame}{Processadores de uso geral}

%     \begin{itemize}
%         \item Dispositivos programáveis;
%         \item Possuem grande número de instruções;
%         \item Pode executar múltiplos processos simultaneamente;
%         \item Menor tempo de desenvolvimento;
%         \item Maior custo unitário.
        
%     \end{itemize}

% \end{frame}

% \begin{frame}{Processadores especializados}

%     \begin{itemize}
%         \item Número de instruções reduzidos;
%         \item Menor custo e poder de processamento;
%         \item Menor custo unitário; 
%         \item Maior tempo de desenvolvimento;
%         \item Podem ser programados;
%     \end{itemize}

    
% \end{frame}


% \begin{frame}{Processadores especializados}

%     \begin{itemize}
%         \item Microcontroladores
            
%             \begin{itemize}
%                 \item CPU, RAM, I/O, UART, I²C e SPI em um mesmo chip;
%                 \item Otimizado para aplicações de controle;
%                 \item Não lida com grande volume de dados ou cálculos complexos;
%             \end{itemize}
            
%         \item \textit{Digital Signal Processors}
            
%             \begin{itemize}
%                 \item Semelhante a microcontroladores;
%                 \item Realiza operações de adição e multiplicação mais eficientemente;
%                 \item Processa grande volume de dados;
%                 \item Otimizado para processamento de sinais;
%             \end{itemize}
            
%     \end{itemize}

    
% \end{frame}




% \begin{frame}{Processadores dedicados}

% \begin{itemize}
%     \item Não programáveis;
%     \item Implementa instruções para uma aplicação em específica;
%     \item Maior custo de desenvolvimento;
% \end{itemize}

% \end{frame}


% \begin{frame}{Processadores dedicados}

%     \begin{itemize}
%         \item Application Specific Integrated Circuit (ASIC)
%         \begin{itemize}
%             \item Circuito integrado de aplicação específica;
%             \item Possui a lógica necessária para execução de tarefas;
%             \item Não permite reconfiguração da lógica implementada;
%         \end{itemize}
        
%         \item Field Programmable Gate Array
        
%             \begin{itemize}
%                 \item Semelhante ao ASIC;
%                 \item Matriz de blocos lógicos reconfiguráveis;
%                 \item Conexões programáveis interligam os blocos da matriz;
%                 \item Permite a reconfiguração da lógica implementada;
%             \end{itemize}
%     \end{itemize}

% \end{frame}


% \begin{frame}{System-On-A-Chip}
%     \begin{itemize}

%     \item Circuito integrado formado por diversos módulos que compõem um sistema computacional;
    
%     \item Visa reduzir o número de CIs utilizados em um sistema embarcado;
    
%     \item Implementa módulos além de CPU, RAM e I/Os;
    
%         \begin{itemize}
%             \item Módulos de comunicação sem fio; 
%             \item Módulos de processamento de sinais.
%         \end{itemize}
    
%     \end{itemize}
% \end{frame}



\begin{frame}{Desafios de Projeto}

    \begin{itemize}
    
        \item Realizar uso eficiente dos recursos computacionais disponíveis;
        
        
        \item Um sistema embarcado deve atingir a dependabilidade;
        
        \begin{itemize}
            \item Segurança da informação;
            \item Confidencialidade;
            \item Operação segura;
            \item Confiabilidade;
            \item Reparabilidade;
        \end{itemize}
        
        
        
        
    \end{itemize}
    
    
    
    

    
\end{frame}





% \subsection{Subseção I}
% \begin{frame}{Criando Blocos}
%     % Blocks styles
%     \begin{block}{Bloco Padrão}
%         Texto do corpo do bloco.
%     \end{block}

%     \begin{alertblock}{Bloco de Alerta}
%         Texto do corpo do bloco.
%     \end{alertblock}

%     \begin{exampleblock}{Bloco de Exemplo}
%         Texto do corpo do bloco.
%     \end{exampleblock}   
% \end{frame}
